% --------------------------------------------------------------
% This is all preamble stuff that you don't have to worry about.
% Head down to where it says "Start here"
% --------------------------------------------------------------

\documentclass[12pt]{article}

\usepackage[margin=1in]{geometry}
\usepackage{amsmath,amsthm,amssymb}
\usepackage{enumitem}

\newcommand{\N}{\mathbb{N}}
\newcommand{\Z}{\mathbb{Z}}
\newcommand{\Q}{\mathbb{Q}}
\newcommand{\R}{\mathbb{R}}

\newtheoremstyle{mysolutionstyle}
{\topsep}
{\topsep}
{\upshape}
{}
{\bfseries}
{:}
{.5em}
{}

\theoremstyle{mysolutionstyle}
\newtheorem*{soln}{Solution}

\setlength{\parindent}{0pt}

\begin{document}

% --------------------------------------------------------------
%                         Start here
% --------------------------------------------------------------

\title{Portfolio Problem XX}%replace XX with the appropriate number and letter, such as 9B.  If there is no letter, just write the number.
\author{My Name \\ %replace with your name
Foundations of Mathematics} %if necessary, replace with your course title

\maketitle







\begin{soln}
Blah, blah, blah.  Here is an example of the \texttt{align} environment:
%Note 2: Inside the align environment, you do not want to use $-signs.  The reason for this is that this is already a math environment. This is why we have to include \text{} around any text inside the align environment.
\begin{align*}
\sum_{i=1}^{k+1}i & = \left(\sum_{i=1}^{k}i\right) +(k+1)\\
& = \frac{k(k+1)}{2}+k+1 & (\text{by inductive hypothesis})\\
& = \frac{k(k+1)+2(k+1)}{2}\\
& = \frac{(k+1)(k+2)}{2}\\
& = \frac{(k+1)((k+1)+1)}{2}.
\end{align*}
\end{soln}


% --------------------------------------------------------------
%     You don't have to fool with anything below this line.
% --------------------------------------------------------------

\end{document}