
\documentclass[12pt]{article}

\usepackage[margin=1in]{geometry}
\usepackage{amsmath,amsthm,amssymb}
\usepackage{enumitem}

\newcommand{\N}{\mathbb{N}}
\newcommand{\Z}{\mathbb{Z}}
\newcommand{\Q}{\mathbb{Q}}
\newcommand{\R}{\mathbb{R}}

\newtheoremstyle{mysolutionstyle}
{\topsep}
{\topsep}
{\upshape}
{}
{\bfseries}
{:}
{.5em}
{}

\theoremstyle{mysolutionstyle}
\newtheorem*{soln}{Solution}

\setlength{\parindent}{0pt}

\begin{document}


\title{Portfolio Problem 1A}%replace XX with the appropriate number and letter, such as 9B.  If there is no letter, just write the number.
\author{Stephen DuVall \\ %replace with your name
Foundations of Mathematics} %if necessary, replace with your course title

\maketitle
% $(\forall m \epsilon \mathbb{N})(m > 2) \rightarrow (2m, m^2 - 1, m^2 + 1)$ is a Pythagorean Triple
\textbf{Conjecture:} If $a$ and $b$ are type 2 integers, then $a^2 + b^2$ is a type to integer as well \\
\textbf{\emph{Type 2 Integer:}} An integer $a$ is said to be type 2 if and only if there exists an integer $m$ such that$a = 3m + 2$
\hfill
\begin{soln} \textbf{Negation:} $(\forall a,b\ \epsilon\ \mathbb{Z}^*)(a^2 + b^2 \text{ is not type 2})$ ...($\mathbb{Z}^*$ is the set of type 2 integers) \\ 
\hfill
Counterexample:
\begin{align*}
	a &= 17	& b &= 23\\
	a &= 3m+2	& b &= 3n + 2\\
	m &= 5	& n &= 7
\end{align*}
\begin{align*}
	a^2 + b^2 &= z\\
	17^2 + 23^2 &= 818 	&\text{\emph{(basic arithmetic)}}\\
	3x + 2 &= 818 		&\text{\emph{(defn. of type 2 int)}}\\
	3x &= 816 \\ 
	x &= 816 / 3 \\
	x &= 272 			&\text{\emph{(z is a type 2 int)}}
\end{align*}
\end{soln}

\end{document}