
\documentclass[12pt]{article}

\usepackage[margin=1in]{geometry}
\usepackage{amsmath,amsthm,amssymb}
\usepackage{enumitem}

\newcommand{\N}{\mathbb{N}}
\newcommand{\Z}{\mathbb{Z}}
\newcommand{\Q}{\mathbb{Q}}
\newcommand{\R}{\mathbb{R}}

\newtheoremstyle{mysolutionstyle}
{\topsep}
{\topsep}
{\upshape}
{}
{\bfseries}
{:}
{.5em}
{}

\theoremstyle{mysolutionstyle}
\newtheorem*{soln}{Solution}

\setlength{\parindent}{0pt}

\begin{document}


\title{Portfolio Problem 2A}%replace XX with the appropriate number and letter, such as 9B.  If there is no letter, just write the number.
\author{Stephen DuVall \\ %replace with your name
Foundations of Mathematics} %if necessary, replace with your course title

\maketitle
% $(\forall m \epsilon \mathbb{N})(m > 2) \rightarrow (2m, m^2 - 1, m^2 + 1)$ is a Pythagorean Triple
\begin{soln} A Pythagorean Triple is a right triangle with sides $(p, q, r)$ where $p < q < r$\\
The Pythagorean theorem says that all right triangles with sides $(a, b, c)$ will equal $a^2 + b^2 = c^2$
\hfill
\textbf{Conjecture:} $(\forall m\ \epsilon\ \mathbb{N})(m > 2) \rightarrow (2m, m^2-1, m^2+1)$ is a Pythagorean Triple
\begin{proof}
\begin{align*}
    (2m, m^2-1, m^2+1) \\
    (2m)^2 + (m^2 - 1)^2 &= (m^2 + 1)^2 \\
    4m^2 + m^4 - 2m^2 + 1 &= m^4 + 2m^2 + 1 \\
    m^4 + 2m^2 + 1 &= m^4 + 2m^2 + 1
\end{align*}
\end{proof}
$(2m, m^2 - 1, m^2 + 1)$ is a Pythagorean Triple for all counting numbers greater than 2.
\end{soln}

\end{document}