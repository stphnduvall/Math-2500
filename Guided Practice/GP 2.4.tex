\documentclass[12pt]{article}

\usepackage[margin=1in]{geometry}
\usepackage{amsmath,amsthm,amssymb}
\usepackage{enumitem}

\newcommand{\N}{\mathbb{N}}
\newcommand{\Z}{\mathbb{Z}}
\newcommand{\Q}{\mathbb{Q}}
\newcommand{\R}{\mathbb{R}}

\newenvironment{theorem}[2][Theorem]{\begin{trivlist}
\item[\hskip \labelsep {\bfseries #1}\hskip \labelsep {\bfseries #2.}]}{\end{trivlist}}
\newenvironment{lemma}[2][Lemma]{\begin{trivlist}
\item[\hskip \labelsep {\bfseries #1}\hskip \labelsep {\bfseries #2.}]}{\end{trivlist}}
\newenvironment{exercise}[2][Exercise]{\begin{trivlist}
\item[\hskip \labelsep {\bfseries #1}\hskip \labelsep {\bfseries #2.}]}{\end{trivlist}}
\newenvironment{corollary}[2][Corollary]{\begin{trivlist}
\item[\hskip \labelsep {\bfseries #1}\hskip \labelsep {\bfseries #2.}]}{\end{trivlist}}
\newenvironment{prchk}[2][Progress Check]{\begin{trivlist}
\item[\hskip \labelsep {\bfseries #1}\hskip \labelsep {\bfseries #2.}]}{\end{trivlist}}
\newenvironment{bgact}[2][Beginning Activity]{\begin{trivlist}
\item[\hskip \labelsep {\bfseries #1}\hskip \labelsep {\bfseries #2.}]}{\end{trivlist}}
\newenvironment{exe}[2][Exercise]{\begin{trivlist}
\item[\hskip \labelsep {\bfseries #1}\hskip \labelsep {\bfseries #2.}]}{\end{trivlist}}

\setlength{\parindent}{0pt}

\begin{document}

% --------------------------------------------------------------
%                         Start here
% --------------------------------------------------------------

\title{Guided Practice 2.4}%replace X with the appropriate number
\author{Stephen DuVall \\ %replace with your name
Foundations of Mathematics} %if necessary, replace with your course title

\maketitle

\begin{bgact}{2.4}
\hfill
\begin{enumerate}
\item $(\forall \ a \ \epsilon \ \mathbb{R}) (a + 0 = a)$ True. For all real numbers $a$, $a + 0 = a$
\item[3.] $\sqrt{x}\  \epsilon\ \mathbb{R}$. I don't know how to check if this is true. I think it is not a statement or whatever
\item[6.] $(\exists x\ \epsilon\ \mathbb{R}) (x^2+1=0)$. False, $x^2$ could never produce a negative number such that adding one would result in $0$.
\item[7.] $(\forall x\ \epsilon\ \mathbb{R})(x^3 \geq x^2)$. False, $(\frac{1}{5})^3 \ngtr (\frac{1}{5})^2$
\item[9.] If $x^2 \geq 1$, then $x \geq 1$. True because $1^2$ is 1. For $x^2$ to be anything greater than 1, $x$ must be greater than 1.
\item[10.] $(\forall x\ \epsilon\ \mathbb{R})(\text{If } x^2 \geq 1 \text{, then } x \geq 1)$. This is the same thing as above but only written symbolically. 
\end{enumerate}
\end{bgact}

\begin{bgact}{2.5}
\hfill
\begin{enumerate}
\item $(\forall x\ \epsilon\ \mathbb{Z})(x \text{ is a multiple of } 2)$.
    \begin{enumerate}
        \item For every integer $x$, $x$ is a multiple of 2.
        \item False. Because the integer $3$ is not a multiple of 2.
        \item For every integer $x$, $x$ is not a multiple of 2.
        \item $(\forall x\ \epsilon\ \mathbb{Z})(x \text{ is not a multiple of } 2)$
        \begin{itemize}
            \item $(\forall x\ \epsilon\ \mathbb{Z})(x \mod 2 \neq 0)$
        \end{itemize}
    \end{enumerate}
\item $(\exists x\ \epsilon\ \mathbb{Z})(x^3 > 0)$.
    \begin{enumerate}
        \item There exists an integer $x$ such that $x^3 > 0$.
        \item True. Because there exist plenty of integers who's cube is greater than 0. $2^3 = 8$
        \item There exists an integer $x$ such that $x^3 < 0$.
        \item $(\exists x\ \epsilon\ \mathbb{Z})(x^3 < 0)$. I don't think this is correct. I need to spend more time and look at how to correctly negate sentences
    \end{enumerate}
\end{enumerate}

\end{bgact}
\end{document}