\documentclass[12pt]{article}

\usepackage[margin=1in]{geometry}
\usepackage{amsmath,amsthm,amssymb}
\usepackage{enumitem}
\usepackage{listings}

\newcommand{\N}{\mathbb{N}}
\newcommand{\Z}{\mathbb{Z}}
\newcommand{\Q}{\mathbb{Q}}
\newcommand{\R}{\mathbb{R}}

\newenvironment{theorem}[2][Theorem]{\begin{trivlist}
\item[\hskip \labelsep {\bfseries #1}\hskip \labelsep {\bfseries #2.}]}{\end{trivlist}}
\newenvironment{lemma}[2][Lemma]{\begin{trivlist}
\item[\hskip \labelsep {\bfseries #1}\hskip \labelsep {\bfseries #2.}]}{\end{trivlist}}
\newenvironment{exercise}[2][Exercise]{\begin{trivlist}
\item[\hskip \labelsep {\bfseries #1}\hskip \labelsep {\bfseries #2.}]}{\end{trivlist}}
\newenvironment{corollary}[2][Corollary]{\begin{trivlist}
\item[\hskip \labelsep {\bfseries #1}\hskip \labelsep {\bfseries #2.}]}{\end{trivlist}}
\newenvironment{prchk}[2][Progress Check]{\begin{trivlist}
\item[\hskip \labelsep {\bfseries #1}\hskip \labelsep {\bfseries #2.}]}{\end{trivlist}}
\newenvironment{bgact}[2][Beginning Activity]{\begin{trivlist}
\item[\hskip \labelsep {\bfseries #1}\hskip \labelsep {\bfseries #2.}]}{\end{trivlist}}
\newenvironment{exe}[2][Exercise]{\begin{trivlist}
\item[\hskip \labelsep {\bfseries #1}\hskip \labelsep {\bfseries #2.}]}{\end{trivlist}}

\setlength{\parindent}{0pt}

\begin{document}

\title{Guided Practice 3.1}
\author{Stephen DuVall \\
Foundations of Mathematics}

\maketitle

\begin{bgact}{3.1}
\hfill
\begin{enumerate}
\item[1.] A nonzero integer $m$ divides an integer $n$ if and only if $(\exists q\ \epsilon\ \mathbb{Z})(n = m \times q)$
    \begin{align*}
    32 &= 4 \times 8   &  -96 &= 8 \times -12
    \end{align*}
\item[2.]
    \begin{align*}
        7 \nmid 10 \qquad 10 &= \_ \times 7 \\  
        8 \nmid 23 \qquad 23 &= \_ \times 8 \\ 
        4 \nmid 15 \qquad 15 &= \_ \times 4 \\  
    \end{align*} 
\item[6.] $\{2, 4, 8\}$,\qquad $\{3, 9, 27\}$,\qquad $\{5, 25, 100\}$
\item[8.] We would assume that $a | b$ and $b | c$
\item[10.] We would be trying to prove that $a | c$
\end{enumerate}
\end{bgact}

\begin{bgact}{3.2}
\hfill
\begin{enumerate}
	\item[1.] Suppose today is Tuesday
		\begin{enumerate} 
			\item[(a)] 3 days from now it will be Friday.
			\item[(b)] 10 days from now it will be Friday.
			\item[(c)] 17 days from now it will be Friday. 24 days from now it will be Friday.
			\item[(d)] $F_{n} = \{24 + 7n\}(n\ \epsilon\ \mathbb{Z} )$ .... $\{31, 38, 45, 52, ...\}$
			\item[(e)] $\{F_{n} - F_{n-1} \}(n\ \epsilon\ \mathbb{N}) = \{7, 7, 7, ...\}$
			\item[(f)] Every number in my list is 7. If I were to pick any non consecutive numbers then they would be a multiple of 7.
		\end{enumerate}
\end{enumerate}
\end{bgact}

\begin{exercise}{3} 
\begin{lstlisting}[language=Python]

for i in range(1999, 0, -23):  # Count down from 1999 by 23
	if ((y - i) % 8 == 0):  # Check if i is divisible by 23
		print(i)  
# 1999
# 1815
# 1631
# 1447
# 1263
# 1079
# 895
# 711
# 527
# 343
# 159 <-- smallest non-negative number where 23 | (1999 - 159)

\end{lstlisting}
\end{exercise}


\hfill
\end{document}
